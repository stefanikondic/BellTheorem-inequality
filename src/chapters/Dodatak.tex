\chapter{Dodatak}
\section{Srednja vrijednost proizvoda komponenti spinova dvaju \v cestica u proizvoljnim pravcima}

Posmatrajmo dvije čestice spina-$1/2$ u singletnom stanju opisanom vektorom:

\begin{equation}
    | 00 \rangle = \frac{1}{\sqrt2}(| \updownarrows \; \rangle - | \downuparrows \; \rangle) \label{eq:singletno_stanje}.
\end{equation}
Neka je $S_k^{(n)}$ operator komponente spina čestice $n \ (n = 1, 2)$ u pravcu k. Da bismo izračunali srednju vrijednost proizvoda komponentni spinova dvaju čestica u proizvoljnim pravcima zadatim ortovima $\vec{a}$ (za prvu česticu) i $\vec{b}$ (za drugu česticu), odredićemo prvo djelovanje proizvoda operatora tih komponenti na stanje (\ref{eq:singletno_stanje}).
Pošto govorimo o proizvoljnim pravcima izaberimo ih tako da se pravac $\vec{a}$ poklapa sa $z$ osom, a pravac $\vec{b}$ neka leži u $xz$ ravni:
\begin{equation*}
    S_a^{(1)} = S_z^{(1)} \quad S_b^{(2)} = cos{\theta} S_z^{(2)} + sin{\theta} S_x^{(2)}.
\end{equation*}

Da bismo odredili djelovanje proizvoda operatora $\hat{S}_a^{(1)}$
i $\hat{S}_b^{(2)}$ na stanje \eqref{eq:singletno_stanje}, potrebno je da znamo kako
operatori $\hat{S}_x^{(n)}$ i $\hat{S}_z^{(n)}$ djeluju na spinska
stanja $|\!\uparrow\;\!\rangle$ i $|\!\downarrow\;\!\rangle$
$n$-te \v cestice. Po\v sto su ova stanja svojstvena stanja
operatora $\hat{S}_z^{(n)}$, imamo
\begin{equation}
\hat{S}_z^{(n)} |\!\uparrow\;\!\rangle = \frac{\hbar}{2}\,
|\!\uparrow\;\!\rangle, \quad \hat{S}_z^{(n)}
|\!\downarrow\;\!\rangle = -\frac{\hbar}{2}\,
|\!\downarrow\;\!\rangle. \label{Sz.spin.st}
\end{equation}
S druge strane, djelovanje operatora $\hat{S}_x^{(n)}$ na spinska
stanja dobijamo polaze\'ci od djelovanja operatora podizanja i
spu\v stanja $\hat{S}_\pm^{(n)} = \hat{S}_x^{(n)} \pm i
\hat{S}_y^{(n)}$ na njih, koje glasi
\begin{equation}
\hat{S}_+^{(n)} |\!\uparrow\;\!\rangle = 0, \quad \hat{S}_-^{(n)}
|\!\uparrow\;\!\rangle = \hbar\, |\!\downarrow\;\!\rangle,
\end{equation}
\begin{equation}
\hat{S}_+^{(n)} |\!\downarrow\;\!\rangle = \hbar\,
|\!\uparrow\;\!\rangle, \quad \hat{S}_-^{(n)}
|\!\downarrow\;\!\rangle = 0,
\end{equation}
odnosno
\begin{equation}
(\hat{S}_x^{(n)} + i \hat{S}_y^{(n)}) |\!\uparrow\;\!\rangle = 0,
\quad (\hat{S}_x^{(n)} - i \hat{S}_y^{(n)}) |\!\uparrow\;\!\rangle
= \hbar\, |\!\downarrow\;\!\rangle,
\end{equation}
\begin{equation}
(\hat{S}_x^{(n)} + i \hat{S}_y^{(n)}) |\!\downarrow\;\!\rangle =
\hbar\, |\!\uparrow\;\!\rangle, \quad (\hat{S}_x^{(n)} - i
\hat{S}_y^{(n)}) |\!\downarrow\;\!\rangle = 0.
\end{equation}
Sabiranjem odvojeno prve dvije i druge dvije jedna\v cine
neposredno slijedi
\begin{equation}
\hat{S}_x^{(n)} |\!\uparrow\;\!\rangle = \frac{\hbar}{2}\,
|\!\downarrow\;\!\rangle, \quad \hat{S}_x^{(n)}
|\!\downarrow\;\!\rangle = \frac{\hbar}{2}\,
|\!\uparrow\;\!\rangle. \label{Sx.spin.st}
\end{equation}

Koriste\'ci rezultate (\ref{Sz.spin.st}) i (\ref{Sx.spin.st})
dalje imamo:

\begin{equation}
    \begin{aligned}
        S_a^{(1)}S_b^{(2)} | 00 \rangle & = S_z^{(1)}\left(cos{\theta} S_z^{(2)} + sin{\theta} S_x^{(2)}\right) \left[\frac{1}{\sqrt2}(| \updownarrows \; \rangle - | \downuparrows \; \rangle)\right]                                                                                                                                                                  \\[1ex]
                                        & = \frac{1}{\sqrt{2}} \left[ S_z^{(1)}|\uparrow\rangle \left( cos{\theta} S_z^{(2)}|\downarrow\rangle + sin{\theta} S_x^{(2)}|\downarrow\rangle \right) - S_z^{(1)}|\downarrow\rangle \left( cos{\theta} S_z^{(2)}|\uparrow\rangle + sin{\theta} S_x^{(2)}|\uparrow\rangle \right) \right]                                     \\[1ex]
                                        & = \frac{1}{\sqrt{2}} \left[ \frac{\hbar}{2}|\uparrow\rangle \left( -cos{\theta}\frac{\hbar}{2}|\downarrow\rangle + sin{\theta} \frac{\hbar}{2}|\uparrow\rangle \right) + \frac{\hbar}{2}|\downarrow\rangle \left( cos{\theta} \frac{\hbar}{2}|\uparrow\rangle + sin{\theta} \frac{\hbar}{2}|\downarrow\rangle \right) \right] \\[1ex]
                                        & = \frac{\hbar^2}{4\sqrt{2}} \left[ -cos{\theta} |\updownarrows\rangle + sin{\theta}|\upuparrows\rangle + cos{\theta} |\downuparrows\rangle + sin{\theta}|\downdownarrows\rangle \right]                                                                                                                                       \\[1ex]
                                        & = \frac{\hbar^2}{4} \left[ - cos{\theta} \frac{1}{\sqrt{2}} \left( |\updownarrows\rangle -  |\downuparrows\rangle \right) + sin{\theta} \frac{1}{\sqrt{2}} \left( |\upuparrows\rangle + |\downdownarrows\rangle \right) \right]                                                                                               \\[1ex]
                                        & = \frac{\hbar^2}{4} \left[ - cos{\theta}|00\rangle + sin{\theta} \frac{1}{\sqrt{2}} \left( | 11 \rangle + | 1 {-1} \rangle \right) \right].
    \end{aligned}
\end{equation}
Prema tome očekivana vrijednost proizvoda spin komponenti je:

\begin{equation}
    \langle  S_a^{(1)}S_b^{(2)} \rangle = \frac{\hbar^2}{4} \langle 00 | \left[ - cos{\theta}|00\rangle + sin{\theta} \frac{1}{\sqrt{2}} \left( | 11 \rangle + | 1 {-1} \rangle \right) \right]
\end{equation}
što se zbog ortonormiranosti stanja $| 00 \rangle,| 11 \rangle,| 10 \rangle,| 1{-1}\rangle$ svodi na:

\begin{equation}
    \langle  S_a^{(1)}S_b^{(2)} \rangle = -\frac{\hbar^2}{4}cos{\theta},
\end{equation}
odnosno

\begin{equation}
    \langle  S_a^{(1)}S_b^{(2)} \rangle = -\frac{\hbar^2}{4} \vec{a} \cdot \vec{b}.
\end{equation}

U svrhu našeg misaonog eksperimenta mjerenje ćemo vršiti u jedinicama $\dfrac{\hbar}{2}$, pa ćemo uzeti da je srednja vrijednost proizvoda komponenti spinova u proizvoljnim pravcima:

\begin{equation}
    P(\vec{a}, \vec{b}) = - \vec{a} \cdot \vec{b}.
\end{equation}