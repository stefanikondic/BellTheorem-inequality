\chapter*{Sažetak}
\sloppy

Belove nejednakosti su relacije među rezultatima mjerenja spina čestica u zapletenom dvočestičnom stanju, izvedene polazeći od hipoteze o postojanju skrivenih varijabli koje održavaju lokalnost u opisu fizičkih procesa.
Ove relacije su formulisane sa ciljem da se ispitaju fundamentalna pitanja u vezi sa tumačenjem kvantne mehanike i razriješi EPR paradoks.
Međutim, eksperimentalni rezultati su pokazali da su Belove nejednakosti narušene i da je standardni kvantnomehanički opis zapletenosti čestica potpuniji, iako podrazumijeva odustajanje od realizma i intuicije
koje pružaju klasična teorija i svako\-dnevno iskustvo.

{\it{Princip lokalnosti}}, koji ima ključnu ulogu u formulaciji EPR paradoksa,
nameće ograničenja na brzinu prenosa informacija između udaljenih dijelova sistema (dvije spregnute čestice, razdvojene nakon interakcije). Belove nejednakosti ukazuju na to da kvantna mehanika nudi nešto što prevazilazi ova ograničenja, sugeriše fenomen {\it{nelokalnosti}}. Ovaj aspekt kvantne mehanike postaje sve relevantniji u kontekstu aktuelnih istraživanja i tehnoloških primjena poput kvantne teleportacije i kvantne informatike.

Nobelova nagrada za fiziku u 2022. godini je bila posvećena upravo istraživanju nelokalnosti u kvantnoj mehanici. Ona odaje priznanja za značajne doprinose naučnika u razumijevanju ovog fenomena, uključujući i eksperimentalna testiranja Belovih nejednakosti. Ova priznanja ne samo da ističu značaj Belovih nejednakosti i njihovu ulogu u oblikovanju našeg razumijevanja kvantne mehanike, već i podstiču dalja istraživanja u ovoj uzbudljivoj oblasti fizike.

{\textbf{Ključne riječi}}: Belove nejednakosti, EPR paradoks, EPRB eksperiment, lokalnost, kvantna zapletenost, kvanta teleportacija, kvantna informatika