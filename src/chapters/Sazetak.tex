\chapter*{Sažetak}
\sloppy

Belova nejednakost je koncept koji proizilazi iz EPR ({\it{Einstein-Podolsky-Rosen}}) paradoksa, koji je postavljen kako bi ispitao fundamentalna pitanja o interpretaciji kvantne mehanike.
Kao potencijalno razrješenje paradoska ispitalo se postojanje skrivenih varijabli koje bi održavale lokalnost u opisu fizičkih procesa i kao rezultat su dobijene Belove nejednakosti.
Međutim, eksperimentalno je pokazano da su Belove nejednakosti narušene i da je kvantnomehančki opis spregnutosti čestica kompletniji, iako podrazumijeva na neki način napuštanje realizma i intuicije koju
nam klasična teorija i svakodnevno iskustvo donose.

{\it{Princip lokalnosti}}, koji je ključan u postavljanju EPR paradoksa,
postavlja ograničenja na brzinu prenosa informacija između udaljenih dijelova sistema (dvije spregnute čestice, razdvojene nakon interakcije). Belove nejednakosti ukazuju na to da kvantna mehanika nudi nešto što prevazilazi ova ograničenja, što ukazuje na fenomen {\it{nelokalnosti}}. Ovaj aspekt kvantne mehanike postaje sve relevantniji u kontekstu aktuelnih istraživanja i tehnoloških aplikacija poput kvantne teleportacije i kvantne kriptografije.

Nobelova nagrada za fiziku u 2022. godini je bila posvećena upravo istraživanju nelokalnosti u kvantnoj mehanici. Ova nagrada priznaje značajne doprinose naučnika u razumijevanju ovog fenomena, uključujući i eksperimentalna testiranja Belovih nejednakosti. Ova priznanja ne samo da naglašavaju značaj Belovih nejednakosti i njihovu ulogu u oblikovanju našeg razumijevanja kvantne mehanike, već i podstiču dalja istraživanja u ovoj uzbudljivoj oblasti fizike.

{\textbf{Ključne riječi}}: Belove nejednakosti, EPR paradoks, EPRB kvantna spregnutost, kvantna zapletenost, sablasno dejstvo na daljinu, kvanta teleportacija, kvantna kriptografija, lokalnost