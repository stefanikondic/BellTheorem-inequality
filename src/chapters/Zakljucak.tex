\chapter*{Zaključak}

O\v cigledno je da u Belovom eksperimentu postoji korelacija u podacima, s tim \v sto tu korelaciju mo\v zemo da vidimo
samo nakon \v sto uporedimo dva seta podataka (nakon \v sto su oba mjerenja zavr\v sena).

U skladu sa tim, odgovor na pitanje: "Da li mjerenje elektrona uti\v ce na sa njim spregnuti pozitron", je {\it{da}}, ali ne u obi\v cnom smislu te rije\v ci.
Da je odgovor negativan, ne bismo mogli objasniti korelacije u podacima.

Osoba koja vr\v si eksperiment, mo\v ze jedino da kontroli\v se da li \' ce izvr\v siti mjerenje ili ne, ali ne mo\v ze uticati na njegov ishod i ti rezultati su prakti\v cno
nasumi\v cni (ako se posmatraju odvojeno od rezulatata eksperimenata sa spregnutom \v cesticom) i ona ne mo\v ze koristiti svoje mjerenje da po\v salje signal osobi
na detektoru spregnute \v cestice (ne mo\v ze natjerati datu \v cesticu da promijeni spin, kao \v sto ni osoba u $X$ ne mo\v ze da uti\v ce na sjenku bube).

Ovo nas navodi da razlikujemo dvije vrste uticaja: "uzro\v cne" vrste i "eteri\v cne" vrste.
Uzro\v cne mo\v emo detektovati vr\v se\' ci mjerenje na samom podsistemu, dok je za eteri\v cne, koje ne prenose informacije ili energiju, jedini dokaz
korelacija u podacima dobijenim eksperimentima na dva razli\v cita podsistema.

Uzro\v cni uticaji se ne mogu \v siriti supreluminalnim brzinama, ali zasad ne postoji razlog za\v sto bi to bilo zabranjeno za eteri\v cne.