\chapter*{Abstract}

Bell's inequality is a concept that stems from the EPR (Einstein-Podolsky-Rosen) paradox, which was formulated to probe fundamental questions regarding the interpretation of quantum mechanics. As a potential resolution to the paradox, the existence of hidden variables maintaining locality in the description of physical processes was explored, resulting in Bell's inequalities. However, experimental evidence has shown that Bell's inequalities are violated, indicating that the quantum mechanical description of particle entanglement is more complete, albeit implying a departure from realism and intuition provided by classical theory and everyday experience.

The principle of locality, crucial in the formulation of the EPR paradox, imposes constraints on the speed of information transfer between distant parts of a system (two entangled particles separated after interaction). Bell's inequalities suggest that quantum mechanics offers something beyond these constraints, pointing to the phenomenon of non-locality. This aspect of quantum mechanics becomes increasingly relevant in the context of current research and technological applications such as quantum teleportation and quantum cryptography.

The Nobel Prize in Physics in 2022. was dedicated to the study of non-locality in quantum mechanics. This recognition acknowledges significant contributions of scientists in understanding this phenomenon, including experimental tests of Bell's inequalities. These results not only underscore the importance of Bell's inequalities and their role in shaping our understanding of quantum mechanics but also stimulate further research in this exciting field of physics.



{\textbf{Keywords}}: Bell's inequalities, EPR paradox, EPRB experiment, quantum entanglement, spooky action at a distance, quantum teleportation, quantum cryptography, locality