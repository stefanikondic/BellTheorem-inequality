\chapter{EPR paradoks}
\section{Originalna formulacija}
EPR paradoks je posljedica misaonog eksperimenta u toku kojeg se desi kolaps talasne funkcije nekog sistema iako nad tim sistemom nije izvršeno mjerenje, a koji je imao za cilj da pokaže da ili kvantna teorija nije kompletna ili dvije opservable opisane nekomutirajućim operatorima ne mogu postojati istovremeno - imati simulatnu realnost. Takođe, konstituent ovog paradoksa je mogućnost pridruživanja više od jedne talasne funkcije prethodno pomenutom sistemu, koja bi trebalo da ga opisuje.
Ajnštaj, Podolski i Rozen su smatrali da bi teorija bila kompletna svaki relevantan element realnosti koji posmatramo mora imati svoj reprezent u fizičkoj teoriji, slobodnije rečeno funkcija preslikavanja realnosti na teoriju mora biti injektivna, ali ne mora biti surjektivna. Primjer za to je talasna funkcija, koju možemo izračunati i "postoji", ali ju ne možemo posmatrati eksperimentalno.
Dakle, premisa ovog misaonog eksperimenta je da ako nešto možemo izračunati i znati bez mjerenja vezano za neku opservablu, onda joj se može pridružiti element realnosti koji joj korespondira.

Misaoni eksperiment sastoji se u sljedećem:
Neka su dati sistemi $A$ i $B$, čije pojedinačne talasne funkcije su nam poznate. Dozvolimo li interakciju ovih sistema u trajanju od neko $\Delta t$ ukupni sistem će se naći u superponiranom stanju.
To stanje je opisano jednačinom koja predstavlja tenzorski proizvod funkcija stanja prvog i drugog sistema prije interakcije:

\begin{equation}
    \Psi(x_1, x_2) = \sum_{n=1}^{\infty} \psi_n(x_2)u_n(x_1),
\end{equation}
gdje su $x_1$, $x_2$ varijable kojima opisujemo prvi i drugi sistem, respektivno (svi njihovi stepeni slobode, ne nužno samo položaj).
Neka su $u_n(x_1)$ svojstvene funkcije prvog sistema koje korespondiraju nekoj opservabli $A$, sa odgovarajućim svojstvenim vrijednostima $a_n$.

Ako potom razdvojimo te sisteme i pretpostavimo da više ne postoji nikakva interakcija, činjenica je da ne možemo znati u kojem stanju se nalazi pojedinačni sistem sve dok ne izvršimo mjerenje.

Pretpostavimo da smo izvršili mjerenje na prvom sistemu i našli da je svojstvena vrijednost $a_k$ opservable $A$. To bi značilo da je korespondirajuća svojstvena funkcija $u_k(x_1)$, te da je drugi sistem u stanju opisanom sa $\psi_k(x_2)$.

Jednačinu koja opisuje superponirano stanje smo mogli posmatrati i kao razvoj talasne funkcije u red sa određenim koeficijentima, koji su igrom slučaja talasne funkcije drugog sistema.

Da smo umjesto opservable $A$ odlučili posmatrati opservablu $B$, talasna funkcija superponiranog stanja u bazisu svojstvenih funkcija $v_n$ pridruženih opservabli $B$ izgledala bi ovako:

\begin{equation}
    \Psi(x_1, x_2) = \sum_{n=1}^{\infty} \phi_n(x_2)v_n(x_1).
\end{equation}
Ako ponovo pretpostavimo da smo izvršili mjerenje na prvom sistemu, samo je ovaj put u pitanju opservabla $B$, prvi sistem će se naći u stanju opisanom sa $v_r(x_1)$ i svojstvenom vrijednosti $b_r$, što bi značilo da je drugi sistem u stanju opisanom sa $\phi_r(x_2)$.

Ono što vidimo iz prethodnog je da drugi sistem u ovom slučaju nema jedinstvenu talasnu funkciju koja ga opisuje. Desila se redukcija talasnog paketa drugog sistema, a da pri tom nikakvo mjerenje nije bilo vršeno nad njim. Takođe, opservable $A$ i $B$ ne moraju nužno komutirati, što nas ostavlja u situaciji da je sistem nejednoznačno opisan i još funkcijama čije postojanje ne može biti simultano.


Iz prethodnog možemo zaključiti ili da opis realnosti dat talasnom funkcijom u kvantnoj mehanici nije kompletan
ili dvije fizičke veličine opisane nekomutirajućim operatorima ne mogu imati simultanu realnost.

\section{Bomova verzija EPR eksperimenta}

Kako bismo bolje opisali smisao EPR paradoksa, razmotrimo Bomovu verziju EPR misaonog eksperimenta, tzv. EPRB eksperiment.\\
Bom posmatra neutralni pion u stanju mirovanja koji se zatim raspada na elektron i pozitron

\begin{equation}
  \pi^0 \rightarrow e^- + e^+.
\end{equation}
S obzirom na to da je pion mirovao, nakon njegovog raspada, elektron i pozitron će se, zbog odr\v zanja impulsa, kretati du\v z istog pravca, ali u suprotnim smjerovima (slika: \ref{fig:pion_decay}).

\begin{figure}[H]
  \[
    \Large{
      \overset{e^-}{\xleftarrow{\hspace{2cm}}}
      \overset{\pi^0}{\bullet}
      \overset{e^+}{\xrightarrow{\hspace{2cm}}}
    }
  \]
  \caption{Bomova verzija EPR misaonog eksperimenta - pion u mirovanju se raspada na elektron i pozitron.}
  \label{fig:pion_decay}
\end{figure}


Kako neutralni pion ima spin jednak nuli, zbog održanja momenta impulsa, ukupni spin elektrona i pozitrona mora takođe biti jednak nuli,
tako da će elektron i pozitron okupirati singletno spinsko stanje:

\begin{equation}
  | \;\! 0 \:\! 0 \:\! \rangle = \frac{1}{\sqrt2}(| \! \uparrow\downarrow \;\! \rangle - | \! \downarrow\uparrow \;\! \rangle). \label{eq:singlet_state}
\end{equation}
Ako potom odlučimo izmjeriti spin elektrona, recimo u pravcu normalnom na pravac kretanja, znamo sa sigurnošću da će spin pozitrona u tom istom pravcu biti suprotan.
Ono što ne možemo znati je koju kombinaciju ćemo dobiti, tj. da li će elektron biti sa spinom gore, a pozitron sa spinom dole, ili obrnuto.

Moguća su različita tumačenja rezultata opisanog misaonog eksperimenta.
Slijedeći stanovište koje su zastupali Ajnštajn, Podolski i Rozen (često nazivano "realističnim"), čestice su odmah u trenutku njihovog nastajanja imale određenu vrijednost spina, samo što kvantna mehanika, kao nekompletna teorija, to nije u mogućnosti da opiše.
Po njima, talasna funkcija nije dovoljna za opis stanja – pored talasne funkcije potrebna je još neka veličina $\lambda$ (ili više njih) da bi se stanje sistema u potpunosti okarakterisalo. Veličina $\lambda$ se obično naziva "skrivena varijabla".

Standardna kvantna mehanika (tzv. "ortodoksno" stanovište), međutim,  podrazumijeva da nijedna čestica nije imala određen spin sve do trenutka mjerenja koje je dovelo do kolapsa talasne funkcije (tj. stanja (\ref{eq:singlet_state})) i trenutno "proizvelo" spin pozitrona, bez obzira koliko on u tom trenutku bio udaljen od elektrona.
Ajnštajn, Podolski i Rozen smatrali su takvo "sablasno dejstvo na daljinu" (Ajnštajnove riječi) apsurdnim. Zaključili su da je ortodoksni stav neodrživ - elektron i pozitron su morali od početka imati dobro definisane spinove, bez obzira da li kvantna mehanika to može da izračuna ili ne.

\section{Princip lokalnosti}

Osnovna pretpostavka na kojoj počiva stanovište Ajnštajna, Podolskog i Rozena je da se nijedan uticaj ne može prostirati brže od svjetlosti. Ovo nazivamo principom lokalnosti. U pokušaju očuvanja ovog principa u okviru standardne kvantne mehanike možemo pretpostaviti da kolaps talasne funkcije nije trenutan već da "putuje" nekom konačnom brzinom. To bi, međutim, dovelo do kršenja zakona o održanju ugaonog momenta.
Naime, ako bismo izmjerili spin pozitrona prije nego što je stigla informacija o kolapsu, vjerovatnoća da pronađemo čestice sa suprotno ili jednako usmjerenim spinovima bila bi ista. Eksperimenti su u tom pogledu nedvosmisleni - takvo kršenje se ne dešava, tj. korelacija spinova je savršena.

Prema tome, princip lokalnosti i kolaps talasne funkcije se međusobno isključuju, a očuvanje ovog prvog je bio glavni argument u traganju za adekvatnom teorijom skrivenih varijabli.