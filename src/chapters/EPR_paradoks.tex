\chapter{EPR paradoks}
\section{Originalna formulacija}
U svom radu pod nazivom {\it{Može li se kvantnomehanički opis fizičke stvarnosti smatrati potpunim?}} iz 1935. godine Ajnštajn, Podolski i Rozen razmatraju pitanje kompletnosti kvantne mehanike kao teorije.
Oni polaze od stava da se neka teorija može smatrati kompletnom ukoliko  svaki relevantan element stvarnosti koji posmatramo ima svoj reprezent u fizičkoj teoriji.
Elementi fizičke stvarnosti ne mogu se odrediti a priori filozofskim razmatranjima, već se moraju pronaći pozivanjem na rezultate eksperimenata i mjerenja. U tom cilju oni uvode sljedeći kriterijum: {\it{"Ako, bez narušavanja sistema, možemo sa sigurnošću (tj. sa vjerovatnoćom jednakom jedinici) da predvidimo vrijednost fizičke veličine, onda postoji element fizičke stvarnosti koji odgovara ovoj fizičkoj veličini."}}


U nastavku ćemo opisati misaoni eksperiment koji su autori razmatrali u svom radu da bi pokazali da pretpostavka o kompletnosti kvantne mehanike dovodi do paradoksa.
Neka su dati sistemi $I$ i $II$, čije pojedinačne talasne funkcije prije uklju\v cenja interakcije su nam poznate. Dozvolimo li interakciju ovih sistema u nekom ograničenom vremenskom intervalu $\Delta t$,
ukupni sistem $(I + II)$ će se nakon toga naći u stanju $\Psi$ koje se, zahvaljujući poznavanju početnih stanja pojedinačnih sistema,
može izračunati rješavanjem (vremenski zavisne) Šredingerove jednačine.
Činjenica je, međutim, da nakon toga ne možemo znati u kojim stanjima se nalaze pojedinačni sistemi $I$ i $II$, čak i po prestanku interakcije, sve dok ne izvršimo mjerenje.

Neka su $u_n(x_1)$ svojstvene funkcije prvog sistema koje odgovaraju nekoj opservabli $A$, sa odgovarajućim svojstvenim vrijednostima $a_n$.
Tada se talasna funkcija $\Psi$ ukupnog sistema po prestanku interakcije može predstaviti u obliku razvoja:

\begin{equation}
    \Psi(x_1, x_2) = \sum_{n=1}^{\infty} \psi_n(x_2)u_n(x_1), \label{eq:talasna_funkcija_ukupnog_sistema_nakon_interakcije}
\end{equation}
gdje su $x_1$, $x_2$ varijable kojima opisujemo prvi i drugi sistem, respektivno (svi njihovi stepeni slobode, ne nužno samo položaj).
Funkcije $\psi_n(x_2)$ u ovom razvoju treba shvatiti kao koeficijente koji stoje uz funkcije $u_n(x_1)$.

Pretpostavimo da smo izvršili mjerenje opservable {\it{A}} na prvom sistemu i kao rezultat dobili svojstvenu vrijednost $a_k$.
Prema tumačenju standardne kvantne mehanike (Kopenhagenska škola) to znači da je došlo do redukcije talasnog paketa (\ref{eq:talasna_funkcija_ukupnog_sistema_nakon_interakcije}), nakon čega se prvi sistem nalazi u stanju opisanim svojstvenom funkcijom $u_k(x_1)$, a drugi sistem u stanju opisanim sa $\psi_k(x_2)$.

Da smo umjesto opservable $A$ odlučili posmatrati opservablu $B$, razvoj talasne funkcije ukupnog sistema u bazisu svojstvenih funkcija $v_n$ pridruženih opservabli $B$ izgledao bi ovako:

\begin{equation}
    \Psi(x_1, x_2) = \sum_{n=1}^{\infty} \phi_n(x_2)v_n(x_1).
\end{equation}
Ako ponovo pretpostavimo da smo izvršili mjerenje na prvom sistemu, samo ovaj put opservable $B$ i dobili rezultat $b_r$, prvi sistem će se naći u stanju opisanom sa $v_r(x_1)$, što bi značilo da je drugi sistem u stanju opisanom sa $\phi_r(x_2)$.

Iz gornje analize proizilazi da, kao posljedica dva različita mjerenja izvršena na prvom sistemu,
drugi sistem može ostati u stanjima sa dvije različite talasne funkcije.
Međutim, pošto u trenutku mjerenja ova dva sistema više ne interaguju, Ajnštajn, Podolski i Rozen smatraju da u drugom sistemu ne može doći do stvarne promjene kao posljedice
bilo čega što se može učiniti sa prvim sistemom.
Dakle, po njima, paradoks se sastoji u tome da je moguće dodijeliti dvije različite talasne funkcije (u našem primjeru $\psi_k$ i $\phi_r$) istoj stvarnosti (tj. drugom sistemu nakon interakcije sa prvim).
Štaviše, opservable $A$ i $B$ ne moraju nužno komutirati, što nas ostavlja u situaciji da je sistem nejednoznačno opisan i još funkcijama koje su svojstvene fizičkim veličinama čije poznavanje ne može biti istovremeno.


Ajnštajn, Podolski i Rozen konačno zaključuju da ili opis realnosti dat talasnom funkcijom u kvantnoj mehanici nije kompletan
ili dvije fizičke veličine opisane nekomutirajućim operatorima ne mogu imati simultanu realnost.

\section{Bomova verzija EPR eksperimenta}

Kako bismo bolje opisali smisao EPR paradoksa, razmotrimo Bomovu verziju EPR misaonog eksperimenta, tzv. EPRB eksperiment.\\
Bom posmatra neutralni pion u stanju mirovanja koji se zatim raspada na elektron i pozitron

\begin{equation}
  \pi^0 \rightarrow e^- + e^+.
\end{equation}
S obzirom na to da je pion mirovao, nakon njegovog raspada, elektron i pozitron će se, zbog odr\v zanja impulsa, kretati du\v z istog pravca, ali u suprotnim smjerovima (slika: \ref{fig:pion_decay}).

\begin{figure}[H]
  \[
    \Large{
      \overset{e^-}{\xleftarrow{\hspace{2cm}}}
      \overset{\pi^0}{\bullet}
      \overset{e^+}{\xrightarrow{\hspace{2cm}}}
    }
  \]
  \caption{Bomova verzija EPR misaonog eksperimenta - pion u mirovanju se raspada na elektron i pozitron.}
  \label{fig:pion_decay}
\end{figure}


Kako neutralni pion ima spin jednak nuli, zbog održanja momenta impulsa, ukupni spin elektrona i pozitrona mora takođe biti jednak nuli,
tako da će elektron i pozitron okupirati singletno spinsko stanje:

\begin{equation}
  | \;\! 0 \:\! 0 \:\! \rangle = \frac{1}{\sqrt2}(| \! \uparrow\downarrow \;\! \rangle - | \! \downarrow\uparrow \;\! \rangle). \label{eq:singlet_state}
\end{equation}
Ako potom odlučimo izmjeriti spin elektrona, recimo u pravcu normalnom na pravac kretanja, znamo sa sigurnošću da će spin pozitrona u tom istom pravcu biti suprotan.
Ono što ne možemo znati je koju kombinaciju ćemo dobiti, tj. da li će elektron biti sa spinom gore, a pozitron sa spinom dole, ili obrnuto.

Moguća su različita tumačenja rezultata opisanog misaonog eksperimenta.
Slijedeći stanovište koje su zastupali Ajnštajn, Podolski i Rozen (često nazivano "realističnim"), čestice su odmah u trenutku njihovog nastajanja imale određenu vrijednost spina, samo što kvantna mehanika, kao nekompletna teorija, to nije u mogućnosti da opiše.
Po njima, talasna funkcija nije dovoljna za opis stanja – pored talasne funkcije potrebna je još neka veličina $\lambda$ (ili više njih) da bi se stanje sistema u potpunosti okarakterisalo. Veličina $\lambda$ se obično naziva "skrivena varijabla".

Standardna kvantna mehanika (tzv. "ortodoksno" stanovište), međutim,  podrazumijeva da nijedna čestica nije imala određen spin sve do trenutka mjerenja koje je dovelo do kolapsa talasne funkcije (tj. stanja (\ref{eq:singlet_state})) i trenutno "proizvelo" spin pozitrona, bez obzira koliko on u tom trenutku bio udaljen od elektrona.
Ajnštajn, Podolski i Rozen smatrali su takvo "sablasno dejstvo na daljinu" (Ajnštajnove riječi) apsurdnim. Zaključili su da je ortodoksni stav neodrživ - elektron i pozitron su morali od početka imati dobro definisane spinove, bez obzira da li kvantna mehanika to može da izračuna ili ne.

\section{Princip lokalnosti}

Osnovna pretpostavka na kojoj počiva stanovište Ajnštajna, Podolskog i Rozena je da se nijedan uticaj ne može prostirati brže od svjetlosti. Ovo nazivamo principom lokalnosti. U pokušaju očuvanja ovog principa u okviru standardne kvantne mehanike možemo pretpostaviti da kolaps talasne funkcije nije trenutan već da "putuje" nekom konačnom brzinom. To bi, međutim, dovelo do kršenja zakona o održanju ugaonog momenta.
Naime, ako bismo izmjerili spin pozitrona prije nego što je stigla informacija o kolapsu, vjerovatnoća da pronađemo čestice sa suprotno ili jednako usmjerenim spinovima bila bi ista. Eksperimenti su u tom pogledu nedvosmisleni - takvo kršenje se ne dešava, tj. korelacija spinova je savršena.

Prema tome, princip lokalnosti i kolaps talasne funkcije se međusobno isključuju, a očuvanje ovog prvog je bio glavni argument u traganju za adekvatnom teorijom skrivenih varijabli.