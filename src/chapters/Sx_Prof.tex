\documentclass[12pt]{article}
\usepackage{graphics,epsf,amsmath}

\topmargin -1cm \oddsidemargin .0cm \evensidemargin .0cm
\textheight 24cm \textwidth 16cm

\begin{document}

\centerline{[ovaj tekst dodati ispred izvodjenja jedna\v cine
(3.2)]}
\bigskip

Da bismo odredili djelovanje proizvoda operatora $\hat{S}_a^{(1)}$
i $\hat{S}_b^{(2)}$ na stanje (3.1), potrebno je da znamo kako
operatori $\hat{S}_x^{(n)}$ i $\hat{S}_z^{(n)}$ djeluju na spinska
stanja $|\!\uparrow\;\!\rangle$ i $|\!\downarrow\;\!\rangle$
$n$-te \v cestice. Po\v sto su ova stanja svojstvena stanja
operatora $\hat{S}_z^{(n)}$, imamo
\begin{equation}
\hat{S}_z^{(n)} |\!\uparrow\;\!\rangle = \frac{\hbar}{2}\,
|\!\uparrow\;\!\rangle, \quad \hat{S}_z^{(n)}
|\!\downarrow\;\!\rangle = -\frac{\hbar}{2}\,
|\!\downarrow\;\!\rangle. \label{Sz.spin.st}
\end{equation}
S druge strane, djelovanje operatora $\hat{S}_x^{(n)}$ na spinska
stanja dobijamo polaze\'ci od djelovanja operatora podizanja i
spu\v stanja $\hat{S}_\pm^{(n)} = \hat{S}_x^{(n)} \pm i
\hat{S}_y^{(n)}$ na njih, koje glasi
\begin{equation}
\hat{S}_+^{(n)} |\!\uparrow\;\!\rangle = 0, \quad \hat{S}_-^{(n)}
|\!\uparrow\;\!\rangle = \hbar\, |\!\downarrow\;\!\rangle,
\end{equation}
\begin{equation}
\hat{S}_+^{(n)} |\!\downarrow\;\!\rangle = \hbar\,
|\!\uparrow\;\!\rangle, \quad \hat{S}_-^{(n)}
|\!\downarrow\;\!\rangle = 0,
\end{equation}
odnosno
\begin{equation}
(\hat{S}_x^{(n)} + i \hat{S}_y^{(n)}) |\!\uparrow\;\!\rangle = 0,
\quad (\hat{S}_x^{(n)} - i \hat{S}_y^{(n)}) |\!\uparrow\;\!\rangle
= \hbar\, |\!\downarrow\;\!\rangle,
\end{equation}
\begin{equation}
(\hat{S}_x^{(n)} + i \hat{S}_y^{(n)}) |\!\downarrow\;\!\rangle =
\hbar\, |\!\uparrow\;\!\rangle, \quad (\hat{S}_x^{(n)} - i
\hat{S}_y^{(n)}) |\!\downarrow\;\!\rangle = 0.
\end{equation}
Sabiranjem odvojeno prve dvije i druge dvije jedna\v cine
neposredno slijedi
\begin{equation}
\hat{S}_x^{(n)} |\!\uparrow\;\!\rangle = \frac{\hbar}{2}\,
|\!\downarrow\;\!\rangle, \quad \hat{S}_x^{(n)}
|\!\downarrow\;\!\rangle = \frac{\hbar}{2}\,
|\!\uparrow\;\!\rangle. \label{Sx.spin.st}
\end{equation}

Koriste\'ci rezultate (\ref{Sz.spin.st}) i (\ref{Sx.spin.st})
dalje imamo

\bigskip
\centerline{[jedna\v cina (3.2) itd.]}


\end{document}
