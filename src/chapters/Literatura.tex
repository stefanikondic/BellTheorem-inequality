\chapter*{Literatura}

\addcontentsline{toc}{chapter}{Literatura}

\begin{enumerate}

\item A. Einstein, B. Podolsky and N. Rosen, \textit{Can
Quantum-Mechanical Description of Physical Reality Be Considered
Complete?}, Phys. Rev. \textbf{47}, 777 (1935)

\item N. Bohr, \textit{Can Quantum-Mechanical Description of
Physical Reality Be Considered Complete?}, Phys. Rev. \textbf{48},
696 (1935)

\item D. Bohm and Y. Aharonov, \textit{Discussion of Experimental
Proof for the Paradox of Einstein, Rosen, and Podolsky}, Phys.
Rev. \textbf{108}, 1070 (1957)

\item J. S. Bell, \textit{On the Einstein Podolsky Rosen paradox},
Physics \textbf{1}, 195 (1964)

\item D. J. Griffiths, \textit{Introduction to Quantum Mechanics}
(Prentice Hall, Upper Sadle River, New Jersey, 1995)

\item B. H. Bransden and C. J. Joachain, \textit{Quantum
Mechanics}, 2nd edition (Pearson, Prentice Hall, Harrlow, England,
2000)

\item The Nobel Prize in Physics 2022, Popular science background:
How entanglement has become a powerful tool (The Royal Swedish
Academy of Sciences, 4 October 2022),
\textsf{https:/\!/www.nobelprize.org/uploads/2022/10/popular-physicsprize2022-3.pdf}

\item Scientific Background on the Nobel Prize in Physics 2022
"for experiments with entangled photons, establishing the
violation of Bell inequalities and pioneering quantum information
science" (The Royal Swedish Academy of Sciences, 4 October 2022),
\textsf{https:/\!/www.nobelprize.org/uploads/2023/10/advanced-physicsprize2022-4.pdf}

\end{enumerate}