\section{Bomova verzija EPR eksperimenta}

Kako bismo bolje opisali smisao EPR paradoksa, razmotrimo Bomovu verziju EPR misaonog eksperimenta, tzv. EPRB eksperiment.\\

Bom posmatra neutralni pion u stanju mirovanja koji se zatim raspada na elektron i pozitron

\begin{equation}
  \pi^0 \rightarrow e^- + e^+.
\end{equation}
S obzirom na to da je pion mirovao, nakon njegovog raspada, elektron i pozitron će se, zbog odr\v zanja impulsa, kretati du\v z istog pravca, ali u suprotnim smjerovima (slika: \ref{fig:pion_decay}).

\begin{figure}[H]
  \[
    \Large{
      \overset{e^-}{\xleftarrow{\hspace{2cm}}}
      \overset{\pi^0}{\bullet}
      \overset{e^+}{\xrightarrow{\hspace{2cm}}}
    }
  \]
  \caption{Bomova verzija EPR misaonog eksperimenta - pion u mirovanju se raspada na elektron i pozitron.}
  \label{fig:pion_decay}
\end{figure}


Kako neutralni pion ima spin jednak nuli, zbog održanja momenta impulsa, ukupni spin elektrona i pozitrona mora takođe biti jednak nuli,
tako da će elektron i pozitron okupirati singletno spinsko stanje:

\begin{equation}
  | 00 \rangle = \frac{1}{\sqrt2}(| \updownarrows \; \rangle - | \downuparrows \; \rangle). \label{eq:singlet_state}
\end{equation}
Ako potom odlučimo izmjeriti spin elektrona, recimo u pravcu normalnom na pravac kretanja, znamo sa sigurnošću da će spin pozitrona u tom istom pravcu biti suprotan.
Ono što ne možemo znati je koju kombinaciju ćemo dobiti, tj. da li će elektron biti sa spinom gore, a pozitron sa spinom dole, ili obrnuto.

Moguća su različita tumačenja rezultata opisanog misaonog eksperimenta.
Slijedeći stanovište koje su zastupali Ajnštajn, Podolski i Rozen (često nazivano "realističnim"), čestice su odmah u trenutku njihovog nastajanja imale određenu vrijednost spina, samo što kvantna mehanika, kao nekompletna teorija, to nije u mogućnosti da opiše.
Po njima, talasna funkcija nije dovoljna za opis stanja – pored talasne funkcije potrebna je još neka veličina $\lambda$ (ili više njih) da bi se stanje sistema u potpunosti okarakterisalo. Veličina $\lambda$ se obično naziva "skrivena varijabla".

Standardna kvantna mehanika (tzv. "ortodoksno" stanovište), međutim,  podrazumijeva da nijedna čestica nije imala određen spin sve do trenutka mjerenja koje je dovelo do kolapsa talasne funkcije (tj. stanja (\ref{eq:singlet_state})) i trenutno "proizvelo" spin pozitrona, bez obzira koliko on u tom trenutku bio udaljen od elektrona.
Ajnštajn, Podolski i Rozen smatrali su takvo "sablasno dejstvo na daljinu" (Ajnštajnove riječi) apsurdnim. Zaključili su da je ortodoksni stav neodrživ - elektron i pozitron su morali od početka imati dobro definisane spinove, bez obzira da li kvantna mehanika to može da izračuna ili ne.

\section{Princip lokalnosti}

Osnovna pretpostavka na kojoj počiva stanovište Ajnštajna, Podolskog i Rozena je da se nijedan uticaj ne može prostirati brže od svjetlosti. Ovo nazivamo principom lokalnosti. U pokušaju očuvanja ovog principa u okviru standardne kvantne mehanike možemo pretpostaviti da kolaps talasne funkcije nije trenutan već da "putuje" nekom konačnom brzinom. To bi, međutim, dovelo do kršenja zakona o održanju ugaonog momenta.
Naime, ako bismo izmjerili spin pozitrona prije nego što je stigla informacija o kolapsu, vjerovatnoća da pronađemo čestice sa suprotno ili jednako usmjerenim spinovima bila bi ista. Eksperimenti su u tom pogledu nedvosmisleni - takvo kršenje se ne dešava, tj. korelacija spinova je savršena.

Prema tome, princip lokalnosti i kolaps talasne funkcije se međusobno isključuju, a očuvanje ovog prvog je bio glavni argument u traganju za adekvatnom teorijom skrivenih varijabli.