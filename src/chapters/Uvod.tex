\chapter{Uvod}

Među fizičkim teorijama kvantna mehanika je svakako jedna od najuspješnijih.
Ovo je potvrđeno i ilustrovano u mnogim njenim aplikacijama.
U stvari, nikada se nije
pokazalo da kvantna mehanika griješi, iako neke specifične primjene mogu biti van
domašaja naših računskih sposobnosti. Ipak, kvantna teorija ima osobine koje djeluju
čudno u poređenju sa klasičnom Njutnovom mehanikom i koje je nekim fizičarima bilo
teško da prihvate.

Jedna od najosnovnijih karakteristika kvantne teorije je njen nedostatak
determinizma. Kada se izvrši jedno mjerenje fizičke veličine $A$, rezultat je jedna od
svojstvenih vrijednosti $a$ odgovarajuće opservable $\hat{A}$.
Međutim, sem kada je sistem u svojstvenom stanju od $\hat{A}$,
nemoguće je predvidjeti za bilo koje pojedinačno mjerenje koja od svojstvenih vrijednosti $a$
će biti dobijena. Sve što se može odrediti je vjerovatnoća dobijanja tih svojstvenih
vrijednosti tako što se mjerenje ponovi mnogo puta na skupu identično pripremljenih
sistema. Ovaj nedostatak determinizma je sasvim drugačiji od bilo čega u klasičnoj
fizici. Istina je, naravno, da se u klasičnoj fizici javljaju mnoge situacije koje
se mogu opisati samo statistički, na primjer kretanje molekula u gasu. Ovaj klasični
indeterminizam proizilazi samo iz našeg nedostatka detaljnog znanja o položajima i
brzinama svakog molekula. Vjeruje se da iako neuočljivo u praksi, u stvari svaki
"klasični" molekul u datom trenutku ima dobro definisan položaj i brzinu, a
rezultati budućih mjerenja položaja i brzine svakog molekula bi se, u principu, mogli
odrediti. Takva razmatranja klasičnih sistema dovela su do pretpostavke da je
kvantna mehanika nekompletna teorija u smislu da postoje druge varijable, nazvane
"skrivene varijable", kojih nismo direktno svjesni, ali koje su potrebne da bi se
sistem u potpunosti odredio. Pretpostavlja se da se ove skrivene varijable ponašaju
na klasičan deterministički način - prividni indeterminizam koji pokazuje
eksperiment proizilazi iz našeg nedostatka znanja o skrivenoj podstrukturi sistema
koji se proučava. Tako su naizgled identični sistemi možda okarakterisani različitim
vrijednostima jedne ili više skrivenih promjenljivih, koje na neki način određuju koje
se svojstvene vrijednosti dobijaju u nekom mjerenju.

Od istorijskog značaja je činjenica da de Broljevo {\it{(Louis de Broglie)}} originalno
tumačenje talasne funkcije spada u klasu teorija skrivenih promjenljivih. On je
pretpostavio da je talasna funkcija fizički realno polje koje se prostire u prostoru
i spregnuto je sa pridruženom česticom koja ima dobro definisan i položaj i impuls.
Sprega između čestice i "pilot talasa" dovodi do uočenih difrakcionih fenomena.
Determinističku teoriju ovog tipa razradio je 1952. godine Dejvid Bom {\it{(David Bohm)}}
koji je bio u stanju da svojom teorijom objasni difrakciju i interferenciju koje se
javljaju pri rasijanju čestica, dobivši potpuno iste rezultate kao one koje daje
kvantna mehanika. Međutim, ovaj model koji sadrži i talase i čestice kao zasebne,
ali povezane entitete, prilično je složen. Za umove većine ljudi on ima još čudnije
karakteristike od onih iz kvantne mehanike, a većina fizičara bi ga odbacila na
osnovu "Okamove oštrice". Možda je najmanje prihvatljiva karakteristika Bomovog
modela, za one koji traže klasični mehanizam u osnovi kvantne teorije, njegova
nelokalnost. Na primjer, u analizi eksperimenta sa dva proreza korišćenjem Bomovog
modela postoji sila koja deluje na česticu koja prolazi kroz jedan prorez i koja se
trenutno mijenja ako se drugi prorez otvori ili zatvori. Takve teorije, u kojima se dejstvo na jednom mjestu prenosi trenutno (ili bar brže od brzine svjetlosti) da bi se
promijenila situacija na drugom, nazivaju se nelokalnim. Kao što ćemo vidjeti, kvantna
mehanika je nelokalna teorija, a nelokalnost se generalno ne smatra prihvatljivom
karakteristikom klasične teorije. Moglo bi se pomisliti da bi se mogla izgraditi
dovoljno genijalna teorija skrivenih promjenljivih, koja je i deterministička i
lokalna. Međutim, Džon S. Bel {\it{(John Stewart Bell)}} je 1965. godine uspio da postavi
uslove koje moraju da zadovolje sve determinističke lokalne teorije.
Ti uslovi, tzv. Belove nejednakosti, su predmet razmatranja ovog diplomskog rada.
Kao što ćemo
vidjeti u nastavku, eksperimenti su pokazali da su ovi uslovi narušeni.

Najpoznatiji fizičar koji je dovodio u pitanje potpunost kvantne teorije bio je
Albert Ajnštajn {\it{(Albert Einstein)}}. Godine 1935, u saradnji sa Borisom Podolskim {\it{(Boris Podolsky)}} i Nejtanom Rozenom
{\it{(Natan Rosen)}}, predložio je sljedeće kriterijume
kao osnovu svake prihvatljive teorije:

\begin{enumerate}
    \item[(1)] Veličine o kojima se govori u teoriji treba da budu "fizički stvarne",
    \item[(2)] Teorija treba da bude lokalna, odnosno da u prirodi nema dejstva na daljinu.
\end{enumerate}

Ajnštajn, Podolski i Rozen su uspjeli da daju primjer kvantnomehaničkog sistema koji
nije zadovoljavao ove uslove i zaključili da je kvantni opis prirode nepotpun.
Slijedeći Boma, istražićemo jednostavniju situaciju od one koju su predložili Ajnštajn
i njegovi saradnici, ali koja pokazuje slične karakteristike. Razmotri\' cemo sistem sa
ukupnim spinom $S = 0$ koji se razdvaja na dvije čestice, $1$ i $2$, svaka sa
spinom $1/2$. Kada su čestice dovoljno razdvojene, mjeri se komponenta spina čestice $1$
paralelna sa nekim pravcem, koji ćemo definisati kao {\it{z}}-osu. Pošto čestica ima spin
$1/2$, dobija se ili rezultat $+\hbar/2$ ili rezultat $-\hbar/2$.

Vidje\' cemo da je čin mjerenja komponente spina čestice $1$ mijenja
rezultat dobijen mjerenjem komponente spina druge čestice. Ova promjena se dešava
trenutno bez obzira koliko su čestice $1$ i $2$ udaljene jedna od druge. Stoga kvantni
opis ne ispunjava uslove $(1)$ i $(2)$. Ova činjenica je poznata kao
Ajnštajn-Podolski-Rozenov (EPR) paradoks. Međutim, Nils Bor {\it{(Niels Bohr)}} je odbacio
ideju da je rezultat paradoksalan, smatrajući da se u uslovu $(1)$ fizička realnost
može odnositi samo na situacije u kojima su eksperimentalni uslovi precizno definisani i sugerišući da to ovdje nije slučaj jer je sistem poremećen od samog početka.

Može se postaviti pitanje šta bi se vidjelo da je spin klasična
varijabla. Ostalo bi neopromijenjeno da bi se dobijale komponente spina čestica $1$ i $2$ koje
su jednake i suprotne, jer je ukupan spin nula. Međutim, ovde bi to bilo zbog toga što bi
vektori spina imali određene vrijednosti i pravce od samog početka kada je stanje stvoreno, a čin mjerenja na čestici $1$ ni na koji način ne bi promijenio stanje čestice
$2$. Tada bi se moglo očekivati da se kvantni rezultati mogu objasniti nekom teorijom skrivenih varijabli. Na prijmer, može postojati klasična skrivena promjenljiva (ili
promjenljive), čija je vrijednost određena kada je kreiran sistem sa nultim spinom i koja je
naknadno odredila eksperimentalne rezultate. Međutim, eksperimentalno kršenje Belove
teoreme, o kojoj ćemo sada raspravljati, pokazuje da je ovo objašnjenje u stvari
netačno.